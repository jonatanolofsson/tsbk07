\section{Grass}
\label{cha:grass}
The grass rendering was the first part of this project to be finalized.
The grass rendering uses a OpenGL shader pipeline with the following steps.
\begin{description}
    \item[Vertex Shader] Calculates points in the terrain where grass strains should be rendered
    \item[Geometry Shader] Generates the grass strains
    \item[Fragment Shader] Paints the grass strains
\end{description}

Most of the work on generating the grass is done in the geometry shader.
The geometry shader is fed with the position of the grass seed, which
is extruded to the full grass strain. As the grass strain is generated - from the bottom-up -
a modeled wind force is applied to bend the grass-strain with the wind.

To give an even more realistic appearance, each grass strain is associated a
random phase offset from the applied wind sinusoid.
This causes each strain to move independently, without removing the sense
that all strains are moved by a common wind.

%~ \fig{0.3}{strain}{Grass strains are built of two parallell strips of triangles to create a 3D-effect.}{strain}

Depending on the distance between the generated grass strain and the camera,
different level of details are selected for the grass, to reduce computational
load on grass strains far away. The higher the level of detail, the more vertices are
used to build the strain. Each strain is built of two parallell triangle strips in an angle
to create a 3D-effect on each grass strain. The texture applied to each strain is also selected
to enhance this 3D-effect, for instance adding shadow in the lower parts of the strain.

%~ \fig{0.8}{grass1}{Grass strains, bent to the applied wind.}{grass1}
