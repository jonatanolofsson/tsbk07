\section{Grass}
\label{cha:grass}
The grass rendering was the first part of this project to be finalized.
The grass rendering uses a OpenGL shader pipeline with the following steps.
\begin{description}
    \item[Vertex Shader] Calculates points in the terrain where grass strains should be rendered
    \item[Geometry Shader] Generates the grass strains
    \item[Fragment Shader] Paints the grass strains
\end{description}

Most of the work on generating the grass is done in the geometry shader.
The geometry shader is fed with the position of the grass seed, which
is extruded to the full grass strain. As the grass strain is generated - from the bottom-up -
a modeled wind force is applied to bend the grass-strain with the wind.

To give an even more realistic appearance, each grass strain is associated a
random phase offset from the applied wind sinusoid.
This causes each strain to move independently, without removing the sense
that all strains are moved by a common wind.

%~ \fig{0.3}{strain}{Grass strains are built of two parallell strips of triangles to create a 3D-effect.}{strain}

Depending on the distance between the generated grass strain and the camera,
different level of details are selected for the grass, to reduce computational
load on grass strains far away. The higher the level of detail, the more vertices are
used to build the strain. Each strain is built of two adjacent triangle strips in an angle
to create a 3D-effect on each grass strain.

A grass like texture is then mapped on this slighty warped rectangle, giving the grass it's green appearance.
However not all grass strains can look the same or the scene would appear artificial.
The tint of the grass can be adjusted individually and every grass is given a different shade of orange mixed with the texture. The strains of grass are also colored darker towards the bottom to simulate the shadow they would cast upon one another in nature.

To make the grass appear shaped like grass and not rectangles, a masking texture is applied to each strain to make the top appear rounded off. Thanks to the discard keyword in GLSL we can get the effect of transparency without disabling the hardware Z depth test.

%~ \fig{0.8}{grass1}{Grass strains, bent to the applied wind.}{grass1}


